
\begin{center}
    \textbf{\Large \sffamily TERMOLOGIA - SCHEDA 2}
\end{center}\vspace{\baselineskip}

\noindent
\centering\begin{tabular}{llll}
    \textbf{Materiale}  & \textbf{Densità}  ($\SI{}{\kilogram/\meter^3}$)  & \textbf{Cal.\ spec.} ($\SI{}{\joule/(\kilogram\cdot\kelvin})$) \\ \hline
    Alcol etilico & $\num{789.3}$ & \num{2430}\\ \hline
    Olio & $\num{800}$ & $\num{1970}$\\ \hline
    Acqua & $\num{1000}$ & $\num{4186}$\\ \hline
    Acqua salata & $\num{1024}$ & $\num{3850}$ \\ \hline
    Glicerina & $\num{1261}$ & $\num{2430}$\\ \hline
    Abete & $\num{700}$ & $\num{1760}$\\ \hline
    Ferro & $\num{7874}$ & $\num{450}$\\ \hline
    Rame & $\num{8960}$ & $\num{386}$\\ \hline
\end{tabular}



\begin{enumerate}

    \item Dentro un bollitore elettrico di potenza $\SI{200}{\watt}$ vengono versati $\SI{30,0}{\centi\liter}$ d'acqua a $\SI{20,0}{\celsius}$. Quanto occorre attendere perché l'acqua cominci a bollire?
    
    \item All'interno di una bacinella di capacità termica trascurabile vengono versati $\SI{1,00}{\liter}$ di alcol etilico a $\SI{20,0}{\celsius}$, $\SI{1,50}{\liter}$ di acqua a $\SI{5,0}{\celsius}$ e $\SI{0,75}{\liter}$ di glicerina a $\SI{8,0}{\celsius}$. La bacinella viene poi posta su un fornelletto di potenza $\SI{1,50}{\kilo\watt}$ per $\SI{10,00}{\second}$. Quale temperatura raggiungono alla fine i tre liquidi?

    \item Una sferetta di ferro di raggio $\SI{2,00}{\cm}$ si trova alla temperatura iniziale di $\SI{50,0}{\celsius}$. Allo scopo di raffreddarla di $\SI{10,0}{\celsius}$, la si immerge all'interno di una bacinella d'acqua a $\SI{5,0}{\celsius}$. Quanti litri d'acqua deve contenere la bacinella (trascurando la capacità termica del recipiente)? Come cambierebbe la risposta se la sferetta fosse ricoperta di uno stato protettivo di rame dello spessore di $\SI{0,50}{\cm}$?
    
    \item Un cilindro rovente di ferro che pesa $\SI{0,250}{\kilogram}$ viene buttato in una bacinella contenente $\SI{2,50}{\liter}$ d'acqua a $\SI{30,0}{\celsius}$. Quanto può essere al massimo la temperatura del cilindro affinché l'acqua non cominci a bollire, se si trascura la capacità termica del recipiente?
    
    \item Durante un afoso pomeriggio d'estate, la temperatura all'interno di una stanza è di $\SI{30,00}{\celsius}$. In una tazza vengono versati $\SI{150}{\gram}$ di latte presi dal frigorifero, all'interno del quale la temperatura è $\SI{4,00}{\celsius}$, e $\SI{100}{\gram}$ di acqua calda a $\SI{90,00}{\celsius}$. Mescolando, il latte annacquato che si ottiene ha una temperatura di $\SI{37,50}{\celsius}$. Si determini la capacità termica della tazza.

    \item Una sfera cava di rame, all'interno della quale è fatto il vuoto, si trova alla temperatura di $\SI{10,0}{\celsius}$, e viene immersa in un recipiente (di capacità termica trascurabile) che contiene $\SI{2,50}{\liter}$ di alcol a $\SI{60,0}{\celsius}$. All'equilibrio, il sistema raggiunge una temperatura di $\SI{58,0}{\celsius}$. Se il raggio esterno della sfera è $r_2=\SI{10,0}{\cm}$, qual è il raggio interno?

\end{enumerate}
