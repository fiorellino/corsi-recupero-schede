
\begin{center}
    \textbf{\Large \sffamily TERMOLOGIA - SCHEDA 1}
\end{center}\vspace{\baselineskip}

\noindent\centering\begin{tabular}{ll}
    \textbf{Materiale}  & \textbf{Densità} ($\SI{}{\kilogram/\meter^3}$) \\ \hline
    Acqua (a \SI{4}{\celsius}) & $\num{1000.0}$\\ \hline
    Alcol etilico (a \SI{20}{\celsius}) & $\num{789.3}$\\ \hline
    Glicerina & $\num{1261}$\\ \hline
\end{tabular}\vspace{\baselineskip}

\noindent\begin{tabular}{ll}
    & \textbf{C.\ dilataz.\ lineare} \\
    \textbf{Materiale} & ($\times 10^{-5}K^{-1}$)\\ \hline
    Ghiaccio (a $\SI{0}{\celsius}$)  & 5.1\\ \hline
    Acciaio & 1.079\\ \hline
    Piombo & 2.799\\ \hline
    Alluminio & 2.336\\ \hline
    Ottone & 1.868\\ \hline
    Rame & 1.666\\ \hline
    Vetro (ordinario) & 0.861\\ \hline
\end{tabular}
\hfill
\noindent\begin{tabular}{ll}
    & \textbf{C.\ dilataz.\ volumica} \\
    \textbf{Materiale} & ($\times 10^{-5}K^{-1}$)\\ \hline
    Acqua  & 20.7\\ \hline
    Alcol etilico & 104\\ \hline
    Cloroformio & 14.0\\ \hline
    Glicerina & 54\\ \hline
    Mercurio & 18.2\\ \hline
    Olio d'oliva & 74\\ \hline
\end{tabular}\vspace{\baselineskip}


\begin{enumerate}

    \item Una barretta di un materiale sconosciuto ha una lunghezza di $\SI{10,000}{\cm}$ alla temperatura di $\SI{100,000}{\kelvin}$, mentre a $\SI{200,000}{\kelvin}$ la sua lunghezza diventa $\SI{10,015}{\cm}$.
    \begin{enumerate}
        \item Quanto è lunga ad una temperatura di $\SI{50,000}{\kelvin}$?
        \item A quale temperatura è lunga $\SI{10,009}{\cm}$?
    \end{enumerate}
    
    \item Mario tiene in mano, mantenendola in orizzontale e trattenendola solo per un'estremità, una lamina costituita da uno strato di piombo (inferiore) e uno di ottone (superiore) sovrapposti. Da quale parte si incurva, se sotto di essa viene accesa una candela?

    \item Un cubo di ottone alla temperatura di $\SI{20}{\celsius}$ ha lato di lunghezza $\SI{30}{\cm}$. Qual è l'incremento della sua superficie totale quando viene riscaldato fino a $\SI{75}{\celsius}$? E quello del suo volume?

    \item Su un righello di alluminio vengono disegnate due circonferenze dello stesso raggio $r_0$; l'interno della prima viene poi rimosso mediante un trapano, così che al suo posto rimanga un foro.
    La temperatura del righello viene poi aumentata di $\SI{16,0}{\kelvin}$.
    \begin{enumerate}
        \item Dei due cerchi, è quello pieno o quello cavo ad avere ora raggio maggiore?
        \item Si calcoli l'incremento percentuale del raggio, del diametro, della circonferenza, e dell'area di ciascuno dei due cerchi.
        \item Se $r_0=\SI{2.0000}{\cm}$, quali sono i raggi finali $r_1$ e $r_1'$ dei due cerchi?
    \end{enumerate}
    
    \item Alla temperatura di $\SI{25,00}{\celsius}$ un'asta di acciaio ha diametro $\SI{3,00000}{\cm}$ e un anello di ottone ha diametro interno di $\SI{2,99200}{\cm}$. A quale temperatura l'asta si infilerà nell'anello?
    
    \item L'alcol etilico ha una densità di $\SI{789.3}{\kilogram/\meter^{3}}$ a $\SI{20}{\celsius}$. Quanto vale la sua densità a $\SI{75}{\celsius}$? 
    
    \item Un vino, a $\SI{5}{\celsius}$, ha una gradazione alcolica del $\SI{12,55}{\percent}$. In altre parole, è formato per il $\SI{12,55}{\percent}$ del suo volume da alcol etilico, e per il restante $\SI{87,45}{\percent}$ del suo volume da acqua. Come cambia il suo tasso alcolico a $\SI{55}{\celsius}$?
    
    \item Un recipiente da $\SI{1,00}{\liter}$ è riempito per metà d'acqua, e per metà d'olio. Qual è la sua capacità termica? E il suo calore specifico (medio)?

    \item Qual è il calore specifico del Campari (bevanda con una gradazione alcolica del $\SI{25,0}{\percent}$)?
\end{enumerate}

