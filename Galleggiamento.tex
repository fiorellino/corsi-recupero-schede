

\begin{center}
    \textbf{\Large \sffamily GALLEGGIAMENTO}
\end{center}\vspace{\baselineskip}


\centering\begin{tabular}{ll}
    \textbf{Materiale}  & \textbf{Densità} ($\SI{}{\kilogram/\meter^3}$) \\ \hline
    Alcol etilico (a \SI{20}{\celsius}) & $\num{789.3}$\\ \hline
    Olio & $\num{800}$\\ \hline
    Acqua (a \SI{4}{\celsius}) & $\num{1000.0}$\\ \hline
    Acqua salata & $\num{1024}$ \\ \hline
    Glicerina & $\num{1261}$\\ \hline
    Abete & $\num{700}$ \\ \hline
    Ferro & $\num{7874}$\\ \hline
    Rame & $\num{8960}$\\ \hline
\end{tabular}

\begin{enumerate}

    \item Un corpo ha peso apparente $\SI{7,00}{\newton}$ nel vuoto e $\SI{5,50}{\newton}$ se è immerso nella glicerina. Quali sono la sua densità e il suo volume?
    
    \item Un cubetto di lato  $\SI{10,0}{\cm}$ galleggia nell'olio rimanendo immerso per un altezza di $\SI{8,75}{\cm}$. Qual è la densità del legno? Nell'acqua, quanto si immergerebbe?

    \item Una zattera di abete a forma di parallelepipedo, con superficie di $\SI{10,0}{\meter^2}$ e spessore di $\SI{0,500}{\m}$, galleggia sul mare. 
    \begin{enumerate}
        \item Quanta parte dello spessore della zattera si trova sott'acqua?
        \item Di quanto si abbassa la zattera, se vi sale un uomo di $\SI{70,0}{\kg}$?
        \item Quanti uomini di $\SI{70,0}{\kg}$ può ospitare al massimo?
    \end{enumerate}
    
    \item Un pezzo di rame, con una cavità interna, ha peso apparente $\SI{3,60}{\newton}$ nel vuoto e $\SI{2,00}{\newton}$ quando è immerso in olio. Qual è il volume della sua cavità interna?

    \item Una sfera cava di ferro di raggio esterno $\SI{60,00}{\cm}$ galleggia quasi completamente sommersa in acqua. Si calcoli il diametro interno del guscio sferico.

    \item Sulla superficie di una piscina galleggia un recipiente cilindrico di rame aperto superiormente. Le pareti del cilindro hanno spessore $\SI{2,00}{\mm}$, il raggio di base è $\SI{500}{\mm}$, l'altezza misura  $\SI{100}{\mm}$.
    \begin{enumerate}
        \item A quale profondità, rispetto al pelo libero dell'acqua, si trova la base del cilindro?
        \item Una falla improvvisamente apertasi sul fondo del recipiente lascia entrare l'acqua alla rapidità di $\SI{3,00}{\milli\liter/\second}$. Dopo quanto tempo affonda?
    \end{enumerate}
    
    \item Un corpo di densità $\SI{950}{\kg/\meter^3}$ viene gettato un bicchiere colmo per metà d'acqua e per metà d'olio. Si calcolino, in rapporto al volume totale del corpo, il volume emerso, il volume immerso in acqua e il volume immerso in olio.
    
    \item Una sferetta di ferro di raggio $\SI{1,00}{\cm}$ è immersa in un lago, ed è collegata ad un legnetto di abete mediante una fune (che supporremo inestensibile, infinitamente sottile e priva di massa). Il legnetto galleggia sulla superficie del lago, impedendo alla sfera di affondare.
    \begin{enumerate}
        \item Quanto vale la tensione della fune?
        \item Quale volume deve avere, come minimo, il legnetto, per riuscire a trattenere la sfera?
    \end{enumerate}
    
    \item Due piccoli parallelepipedi di due diversi materiali sconosciuti, alti ambedue $\SI{10,0}{\cm}$, galleggiano su una vasca da bagno, immersi rispettivamente per il $\SI{30,0}{\percent}$ e per il $\SI{90,0}{\percent}$ della propria altezza. Nella vasca si versa ora dell'olio, fino a formare uno strato di spessore $\SI{20}{\cm}$ sovrapposto all'acqua. Di quanto si spostano verticalmente il primo e il secondo parallelepipedo?
\end{enumerate}