

\begin{center}
    \textbf{\Large \sffamily TERMOLOGIA - SCHEDA 3}
\end{center}\vspace{\baselineskip}

\noindent
\centering\begin{tabular}{llll}
    \textbf{Materiale}  & \textbf{Densità}  ($\SI{}{\kilogram/\meter^3}$)  & \textbf{Cal.\ spec.} ($\SI{}{\joule/(\kilogram\cdot\kelvin})$) \\ \hline
    Alcol etilico (liquido) & $\num{789.3}$ & \num{2430}\\ \hline
    Vapor d'acqua & Variabile & $\num{1996}$\\ \hline
    Acqua & $\num{1000}$ & $\num{4186}$\\ \hline
    Ghiaccio & $\num{920}$ & $\num{2108}$\\ \hline
    Azoto gassoso & Variabile & $\num{1040}$\\ \hline
    Azoto liquido & $\num{808.4}$ & $\num{2000}$ \\ \hline
    Rame & $\num{8960}$ & $\num{386}$\\ \hline
\end{tabular}\vspace{\baselineskip}

\centering\begin{tabular}{lrrrr}
    \textbf{Materiale}  & \textbf{T fus.} ($\SI{}{\celsius}$)  
    & \textbf{T vap.} ($\SI{}{\celsius}$) 
    &  \textbf{Lat.\ fus.}  ($\SI{}{\kilo\joule/\kilogram}$)
    &  \textbf{Lat.\ vap.} ($\SI{}{\kilo\joule/\kilogram}$)
    \\ \hline
    Acqua & $\num{0,00}$ &  $\num{100,00}$
    & $\num{373}$ & $\num{2256}$\\ \hline
    Alcol etilico & $\num{-114.14}$ & $\num{78.37}$ 
    & $\num{109}$ & $\num{838}$\\ \hline
    Azoto & $\num{-210.00}$ &  $\num{-195.80}$ 
    & $\num{26}$ & $\num{199}$\\ \hline
\end{tabular}

\begin{enumerate}

    \item Una scatola ermetica contenente $\SI{1,00}{\kg}$ di vapor d'acqua a $\SI{130,0}{\celsius}$ viene messa in un congelatore, il quale mantiene al proprio interno una temperatura di $\SI{-20,0}{\celsius}$. Quanto calore avrà scambiato il contenuto del recipiente con l'ambiente al raggiungimento dell'equilibrio?

    \item Quanti cubetti di ghiaccio da $\SI{1,50}{\centi\meter}$ di lato (estratti dal congelatore, ove vengono conservati alla temperatura di $\SI{-20,0}{\celsius}$) vanno sciolti in $\SI{1,00}{\liter}$ d'acqua a $\SI{25,0}{\celsius}$, se si vuole raffreddarla di $\SI{10,0}{\celsius}$?
    
    \item Una sferetta di rame di massa $\SI{2,00}{\kilo\gram}$ viene immersa in un bicchiere contenente $\SI{30,0}{\gram}$ d'acqua a $\SI{30,0}{\celsius}$.
    \begin{enumerate}
        \item Quale temperatura deve avere, come minimo, la sferetta, affinché tutta l'acqua contenuta nel bicchiere si vaporizzi?
        \item Quale temperatura deve avere, al massimo, la sferetta, affinché tutta l'acqua contenuta nel bicchiere si solidifichi?
    \end{enumerate}
    
    \item Un cubetto di ghiaccio a $\SI{-20,0}{\celsius}$, di massa $\SI{10,0}{\gram}$, viene immerso in un bicchiere contenente $\SI{50,0}{\gram}$ d'alcol etilico a $\SI{15,0}{\celsius}$.
    \begin{enumerate}
        \item Si descriva esaustivamente lo stato del sistema al raggiungimento dell'equilibrio termico.
        \item Come cambia la risposta se la massa di etanolo è di $\SI{500}{\gram}$? E se è di $\SI{5,00}{\gram}$?
    \end{enumerate}
    
    \item ($\star$) Un recipiente (del quale trascuriamo la capacità termica) contiene $\SI{50,00}{\liter}$ d'acqua a $\SI{25,0}{\celsius}$, alla quale vengono miscelati $\SI{1,50}{\liter}$ di azoto liquido a $\SI{-200,0}{\celsius}$. Si descriva esaustivamente lo stato del sistema al raggiungimento dell'equilibrio termico.
\end{enumerate}
